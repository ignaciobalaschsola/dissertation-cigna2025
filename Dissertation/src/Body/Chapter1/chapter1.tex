\chapter{Introduction}

\section{Context \& Motivation}

In today’s competitive insurance landscape, data-driven decision-making is paramount. Cigna Healthcare Spain has traditionally specialized in the business-to-business (B2B) sector, providing comprehensive insurance solutions to corporations for their employees. This focus has established Cigna as a leader in the Spanish market, renowned for its expertise and excellence in corporate health insurance. In addition to its core B2B offerings, Cigna also caters to individual clients, many of whom are former beneficiaries transitioning from corporate plans. Recognizing the growth potential in this segment, Cigna aims to deepen its understanding of individual policyholder trends to enhance its services and market reach.

To achieve this, Cigna is developing Business Intelligence (BI) dashboards tailored for the sales and executive departments. These dashboards are designed to provide clear visualizations of policyholder trends, focusing on the evolution of the insured population without emphasizing financial metrics. By leveraging these insights, the company can better understand customer demographics, preferences, and behaviors, leading to more informed strategic decisions.

\section{Problem Statement}

Currently, Cigna lacks a centralized system for visualizing and analyzing policyholder data, resulting in fragmented insights and delayed decision-making. The absence of comprehensive BI dashboards hinders the ability of the sales and executive teams to promptly identify trends in the insured population. Furthermore, without integrating expense data, there's a limited understanding of the correlation between policyholder characteristics and incurred costs, impeding accurate expense forecasting and risk assessment.

\section{Research Objectives}

The primary objectives of this dissertation are:

\begin{enumerate}
    \item Develop interactive BI dashboards using Tableau to visualize key policyholder metrics, such as demographic distributions, policy uptake trends, and retention rates.
    \item Collaborate with the finance department to integrate expense data, enabling a comprehensive analysis of the relationship between policyholder attributes and associated costs.
    \item Utilize Machine Learning models, implemented in Python with scikit-learn, to predict individual policyholder expenses based on demographic and behavioral data.
    \item Provide actionable insights to assist in strategic decision-making, policy design, and targeted marketing efforts.
\end{enumerate}

\section{Scope \& Limitations}

This study focuses on individual policyholders within Cigna Healthcare Spain, excluding corporate and group policies. The BI dashboards will concentrate on non-financial metrics related to the insured population, while the integration of expense data will be conducted in collaboration with the finance department. The ML models for expense prediction will be trained on historical data; however, their predictive accuracy may be limited by data quality and the inherent variability of healthcare expenses. Additionally, the deployment of these models will be static, providing a set of parameters for the team, without real-time adaptation or continuous learning capabilities.

\section{Thesis Structure}

This dissertation is structured as follows:

\begin{itemize}
    \item \textbf{Chapter 1: Introduction} – Outlines the background, motivation, problem statement, objectives, scope, and structure of the dissertation.
    \item \textbf{Chapter 2: Literature Review} – Examines existing research on BI applications in the insurance industry and the use of ML for expense prediction.
    \item \textbf{Chapter 3: Methodology} – Describes the data collection process, design and development of BI dashboards, and the implementation of ML models.
    \item \textbf{Chapter 4: Implementation} – Details the technical aspects of developing the BI dashboards and integrating ML models.
    \item \textbf{Chapter 5: Results and Analysis} – Presents the findings from the BI dashboards and evaluates the performance of the ML models.
    \item \textbf{Chapter 6: Discussion and Future Work} – Interprets the results, discusses limitations, and suggests directions for future research.
    \item \textbf{Chapter 7: Conclusion} – Summarizes the key contributions and implications of the study.
\end{itemize}