\begin{center}
    {\large \textbf{Abstract}} \vspace{2cm}
\end{center}

Efficient data management and analysis are critical in the insurance sector for optimizing decision-making and operational efficiency. 

This project, developed at Cigna Healthcare Spain, focuses on enhancing policyholder segmentation through Business Intelligence (BI) and Machine Learning (ML) techniques. By leveraging SQL Server, Tableau, and Python, the project aims to improve customer classification, refine pricing strategies, and optimize marketing efforts.

The methodology involves data preprocessing using ETL techniques, exploratory data analysis (EDA), and unsupervised machine learning for segmentation. 

The final deliverables include interactive Tableau dashboards for real-time insights and an ML-trained model for policyholder expense prediction.

This project aligns with Sustainable Development Goals (SDGs) by promoting efficient resource allocation in health insurance (SDG 3), fostering technological innovation (SDG 9), and improving operational efficiency (SDG 12). 

The results are expected to enhance decision-making at Cigna while setting a foundation for future AI-driven improvements in the insurance industry.